\capitulo{4}{Técnicas y herramientas}

\section{Técnicas}

\subsection{\textit{Scrum}}
La metodología \textit{Scrum} es una metodología de desarrollo ágil, en la que se aplican buenas prácticas y procesos con la finalidad de obtener la mejor calidad posible en el producto final buscado.\newline
Esta metodología está pensada para un equipo de trabajo, con el fin de organizarse y trabajar más eficientemente obteniendo los mejores resultados posibles. En mi caso trataré de simular una metodología \textit{Scrum} ya que el trabajo le realizaré yo solo, pero la intención es dividir dicho trabajo en \textit{sprints} que contendrán dentro las tareas a realizar.\newline
Como acabo de mencionar, esta metodología se basa en crear \textit{sprints}, dentro de los que se crearán las tareas a realizar (\textit{issues}). Para establecer estas tareas dentro de cada sprint, se realiza una planificación, en la que se deja claro los objetivos a conseguir y el tiempo para cada uno de ellos. 
Se realizan revisiones para controlar los avances de los sprint planificados y llevar un control sobre el proyecto.

\subsubsection{\textit{GitHub}}
Para el control de versiones de este proyecto he utilizado \textit{GitHub}, que es un repositorio en línea que emplea \textit{Git}. Así podemos guardar los cambios que vamos realizando en el proyecto y controlar que se cumplan las tareas en el tiempo indicado.

\section{Herramientas}
\subsection{\textit{Visual Studio Code}}
Editor y compilador de código fuente. Incluye y utilizamos el soporte que ofrece para la depuración y para el control integrado de \textit{Azure Web Services}.

\subsection{\textit{Flask}}
\textit{Flask} es un \textit{framework} escrito en \textit{Python} que permite crear aplicaciones web rápidamente y con un mínimo número de líneas de código.

\subsection{\textit{Jinja2}}
\textit{Jinja2} es un motor de plantillas web para aplicaciones desarrolladas en \textit{Python}. Permite que \textit{Flask} pueda hacer uso de los contenidos de las plantillas \textit{HTML}.

\subsection{Documentación}
Como herramienta para realizar la documentación se ha escogido \textit{LaTeX}, está diseñado para crear documentos con una alta calidad tipográfica. Como editor de \textit{LaTeX} he utilizado \texit{Overleaf}

\subsection{Outlook API (Microsoft Graph) }
Para realizar este proyecto se ha utilizado la \texit{API} de \texit{Outlook} para manejar los calendarios, que equivalen a las aulas que se proporcionan al usuario para reservar. Para tener acceso a las herramientas de esta API, usé el punto de conexión Microsoft Graph ya que ofrece más servicios y características siguiendo la recomendación de Microsoft (\url{https://docs.microsoft.com/es-ES/outlook/rest/compare-graph}).
En concreto para este proyecto he utilizado distintas funciones de esta \texit{API}:
\begin{itemize}
    \item \texbf{Microsoft graph} para realizar las llamadas a esta \texit{API} y recoger o escribir datos en los calendarios, y modificar y crear dichos calendarios.
    \item \texbf{Outlook Login} para el inicio de sesión mediante una cuenta de \texit{Outlook}.
\end{itemize}
