\capitulo{2}{Objetivos del proyecto}

En esta sección se enumeran los objetivos que se tratan de abarcar en el proyecto.

\subsection{Objetivos principales}

\begin{itemize}
	\item Desarrollar una aplicación web en la que los usuarios que dispongan de los permisos necesarios puedan reservas aulas.
	\item Mostrar al usuario que acceda a la aplicación web las franjas horarias reservadas y libres de cada aula.
	\item Permitir al administrador la creación de nuevos aulas, donde se podrán agregar eventos.
	\item Posibilidad para el administrador de modificar y borrar la información de las aulas.
	\item Filtrar las aulas según características para una mayor facilidad a la hora de reservar.
	\item Permitir el inicio y cierre de sesión mediante Outlook.
	\item Avisar de la reserva realizada al profesor que la solicitó.
\end{itemize}

\subsection{Objetivos técnicos}
\begin{itemize}
	\item Desarrollar una aplicación web cliente servidor mediante Flask.
	\item Utilizar la API de Outlook para el inicio de sesión y el uso de los calendarios de cada aula.
	\item Utilizar el sistema de control de versiones de Git.
	\item Utilizar ZenHub dentro de GitHub para gestionar el proyecto.
	\item Desplegar la aplicación web y la base de datos en la nube.
\end{itemize}