\capitulo{5}{Aspectos relevantes del desarrollo del proyecto}

En este apartado mostraré los aspectos más interesantes a la hora de desarrollar el proyecto, con esto me refiero a problemas surgidos, los caminos tomados para avanzar, el desarrollo del diseño y de la funcionalidad.

\subsection{Comienzo del proyecto}
Este proyecto me pareció útil e interesante, ya que se centra en el desarrollo web de una aplicación bastante realista. El desarrollo de una aplicación web no ha sido un tema que se haya visto mucho en la carrera por lo que me parecía interesante ya que en las prácticas trabajé un poco con ello y me gustó, por lo que me decidí a escoger este trabajo.   

\subsection{Preparación}
Como ya he comentado en la sección anterior, no tenía los conocimientos suficientes para el desarrollo de esta aplicación web, sobretodo tratando con lenguaje Python, exactamente con el \textit{framework Flask}. Sobre \textit{HTML} y \textit{JavaScript} ya estaba más relacionado.También tuve que tratar con la Microsoft Graph, desde dónde se gestionarían los eventos y calendarios.\newline
Para aprender sobre estos temas antes de comenzar seguí los siguientes tutoriales:\newline

\begin{itemize}
\item \href{https://blog.miguelgrinberg.com/post/the-flask-mega-tutorial-part-i-hello-world}{The Flask Mega Tutorial}\cite{flaskMegaTutorial}
\item \href{https://j2logo.com/tutorial-flask-espanol/}{J2logo}\cite{j2logo}
\end{itemize}
Cuanto más avanzaba el proyecto, más herramientas y distintas técnicas hacían falta, por lo que se tuvo que usar mucha documentación para comprender los funcionamientos de estas, las más consultadas han sido:
\begin{itemize}
\item \href{https://docs.microsoft.com/es-es/graph/api/overview?view=graph-rest-1.0}{Microsoft Graph}\cite{microsoftGraph}
\item \href{https://flask.palletsprojects.com/en/1.1.x/}{Documentación Flask}\cite{Flask}
\item \href{https://getbootstrap.com/docs/4.5/getting-started/introduction/}{Bootstrap}\cite{Bootstrap}
\end{itemize}

\subsection{Metodología del proyecto - Resumen}
Para comenzar el proyecto en Flask se necesitó tener instalado Visual Studio Code desde donde programaríamos nuestra web app, para ello hay que realizar unas instalaciones de requisitos que iban a ser necesarias para el desarrollo del proyecto desde el VSCode:
\begin{enumerate}
    \item Creación de un entorno virtual, para aislar las dependencias que requiere el proyecto de las ya existentes.
    \item Pylint como verificador del código que escribimos.
    \item Git Tortoise para facilitar las subidas a GitHub.
    \item Instalar extensiones de VSCode.
    \item Registrar la aplicación en Azure Active Directory.
\end{enumerate}
El primer punto a comentar es el uso de la API de Outlook, como he comentado accediendo a ella mediante Microsoft Graph, pero para poder acceder a los datos de Outlook y crear un inicio de sesión, primeramente tuve que registrarla en Azure Active Directory:
\begin{itemize}
    \item Acceder al centro de administración de Azure Active Directory.
    \item Acceder al apartado registro de aplicaciones y crear un nuevo registro.
    \item Una vez ahí dentro, se pedirán rellenar unos campos con el nombre de la aplicación, cuentas que accederán y lo más importante, la dirección URL de redirección al iniciar sesión mediante Outlook.
    \item Una vez se ha creado, recogemos el ID de la aplicación obtenido, que lo utilizaremos para obtener el secreto del cliente, que solamente se mostrará una vez. Para obtenerlo accedemos a certificados y secretos, se escribe un nombre y elige una duración y se obtiene dicho secreto.
\end{itemize}
Esto está especificado en el \href{https://docs.microsoft.com/es-es/graph/tutorials/python}{tutorial de Microsoft}\cite{pythonMicrosoftGraph}. \newline
Una vez registrada la aplicación tuve que seguir una serie de pasos para incluirlo en el código y hacer que funcionara:
\begin{itemize}
    \item Agregar al código la autenticación de Azure que hemos conseguido en los pasos anteriores (ver figura \ref{fig:AzureAutenticacion}) .
    \imagen{AzureAutenticacion}{Código recibido de Azure Active Directory}
    \item Agregar al código los permisos de Outlook que se van a proporcionar al usuario que acceda.
    \item Crear unas funciones con llamadas a la API REST para obtener un token de inicio de sesión y otro para refrescarle. Esto servirá para cada sesión de cada usuario diferente.
\end{itemize}
Una vez se creó este primer código que permitía acceder mediante el inicio de sesión de Outlook a la página que especificamos en Azure Active Directory como URL de redirección, pude empezar a trabajar con llamadas a la API REST para obtener información sobre los calendarios del usuario y sus eventos.\newline
Más adelante vi que nos facilitaría la vida el uso de calendarios compartidos ya que todos serían gestionados desde una cuenta administrador en Outlook y estos pueden ser compartidos con los permisos deseados a los propietarios de las aulas, que serán los encargados de realizar reservas sobre dichos calendarios. De esta manera generamos calendarios globales, que equivalen a aulas, y están disponibles para todos los usuarios con los que se compartan. El problema de ésto lo veremos más adelante, ya que compartir estos calendarios tiene que ser una función a realizar por el administrador manualmente.\newline
Un aspecto relevante del proyecto fue la elección del punto de enlace de Microsoft Graph en vez del extremo de la API de Outlook, ya que ofrece más servicios y características. Elección tomada según la documentación de Microsoft \url{https://docs.microsoft.com/es-ES/outlook/rest/compare-graph}\cite{microsoftGraphCompare}. Primeramente se estuvo trabajando con el punto de conexión de Outlook en la v2.0, ya que parecía más nuevo y posiblemente mejor, pero me fui encontrando errores a la hora de utilizarlo y falta de características, lo que me hizo retroceder y perder tiempo en esto, teniendo que buscar una solución y finalmente encontrando que cambiar los puntos de conexión a Microsoft Graph era la mejor.\newline

Durante el desarrollo fueron surgiendo problemas y diferencias en el uso de las API con el tutorial de Microsoft, ya que está diseñado para Django y no para Flask, aquí tuve que decidir si cambiar el comienzo del proyecto y empezar de nuevo con Django para tener este aspecto solucionado o buscar la alternativa en el lenguaje de Flask, mi idea principal era realizar el proyecto en Flask ya que tenía ganas de aprenderlo pues lo tenía entendido como más rápido para el desarrollo, más fácil de aprender y con alcances del proyecto similares, por lo que la solución tomada fue adaptar el código, esto implicó cambios en el código en el que me fijaba y me obligaba a buscar una adaptación de distintos características que tenían que ser usadas. Por ejemplo, para obtener el JWT (JSON Web Token), no nos sirve la función proporcionada ya que se necesitan diferentes atributos, mediante un identificador de token  proporcionado por una llamada a la API REST en este caso.\newline
Como comentaba anteriormente, un error que no se ha podido solventar es automatizar el compartir calendarios desde la aplicación web, ya que no proporciona esta funcionalidad la API REST de Outlook. Se estuvo buscando en la documentación de la que obtenemos todas las funcionalidades de dicha API REST y lo único que encontramos, fue una funcionalidad en fase Beta aún parte de ella aún y la funcionalidad que nos ofrece es la de gestionar los permisos de un calendario compartido con algún usuario, como vemos en la cita textual de esta página \textit{Antes de que se pueda aplicar el uso compartido del calendario o la delegación, el propietario envía una invitación a una persona con la que compartir o delegado, y dicha persona o delegado acepta la invitación}\cite{compartirCalendario}. Investigando y dando vueltas a este problema, se llegó a la conclusión de que la única forma de compartirlos es manualmente desde la cuenta del administrador.\newline
Otro aspecto a tener en cuenta fue la decisión de subir el proyecto a Azure, ya que se había registrado la aplicación en Azure Active Directory y facilitaba una base de datos y permitía alojar la aplicación web sin coste.\newline
En cuanto al trabajo con Azure, comentar que para realizar la subida tuve que reestructurar el proyecto según lo que Microsoft establece al trabajar en un proyecto que utiliza VSCode y Flask.
\begin{itemize}
    \item \href{https://github.com/microsoft/python-sample-vscode-flask-tutorial}{Estructura VSCode y Flask (Documentación microsoft)}\cite{pythonSample}
    \item \href{https://code.visualstudio.com/docs/python/tutorial-flask}{Estructura VSCode y Flask (Documentación Visual Studio)}\cite{flaskTutorialVSCode}
\end{itemize}
Esto conllevó errores en las partes del código que se importaban al cambiar la carpeta raíz desde VSCode. Siguiendo las mismas especificaciones también generé el fichero \textit{launch} para poder ejecutar desde local mediante el debug.\newline
Una vez que el proyecto estaba estructurado seguí los pasos del tutorial que proporciona Miscrosoft para realizar la subida de la aplicación web a los servidores de Azure. \href{https://docs.microsoft.com/es-es/azure/developer/python/tutorial-deploy-app-service-on-linux-01}{Despliegue en Azure}\cite{deployVSCode}.\newline

Otro aspecto a remarcar es la conexión con la base de datos, al subir todo a los servicios de Azure, la conexión ya no se puede realizar como al principio, que se hacía con la base de datos local de MySQL, ahora se realiza una conexión de pyodbc especificando el servidor, la base de datos, el usuario, contraseña y driver a utilizar desde Azure. Esto no permitirá que se ejecutado en cualquier servidor local como se verá más adelante en la explicación del despliegue. Tuve que modificar las consultas que realizaba ya que no serían desde el mismo cursor ni la misma conexión y el lenguaje de consulta cambia de MySQL al utilizarlo en SQL Database proporcionado por Azure.\newline
Como última anotación, comentar que la parte de la vista para el cliente de la wep app y los estilos que daría a la página los dejo para el final, cuando las funcionalidades estén completadas se mejorará la visualización para el cliente.\newline




