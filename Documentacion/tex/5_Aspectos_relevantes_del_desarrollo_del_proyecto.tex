\capitulo{5}{Aspectos relevantes del desarrollo del proyecto}

En este apartado mostraré los aspectos más interesantes a la hora de desarrollar el proyecto, esto hace referencia a problemas surgidos, los caminos tomados para avanzar, el desarrollo de la funcionalidad.

\subsection{Preparación}
Como ya se ha comentado en la sección anterior, no se disponían de los conocimientos suficientes para el desarrollo de esta aplicación web, sobretodo tratando con lenguaje Python, exactamente con el \textit{framework Flask}. Sobre \textit{HTML} y \textit{JavaScript} ya estaba más relacionado.También se tuvo que tratar con Microsoft Graph, desde dónde se gestionarían los eventos y calendarios.\newline
Para aprender sobre estos temas antes de comenzar seguí los siguientes tutoriales:\newline

\begin{itemize}
\item \href{https://blog.miguelgrinberg.com/post/the-flask-mega-tutorial-part-i-hello-world}{The Flask Mega Tutorial}\cite{flaskMegaTutorial}
\item \href{https://j2logo.com/tutorial-flask-espanol/}{J2logo}\cite{j2logo}
\end{itemize}
Cuanto más avanzaba el proyecto, más herramientas y distintas técnicas hacían falta, por lo que se tuvo que usar mucha documentación para comprender los funcionamientos de estas, las más consultadas han sido:
\begin{itemize}
\item \href{https://docs.microsoft.com/es-es/graph/api/overview?view=graph-rest-1.0}{Microsoft Graph}\cite{microsoftGraph}
\item \href{https://flask.palletsprojects.com/en/1.1.x/}{Documentación Flask}\cite{Flask}
\item \href{https://getbootstrap.com/docs/4.5/getting-started/introduction/}{Bootstrap}\cite{Bootstrap}
\end{itemize}

\subsection{Puesta en marcha del entorno de producción y pruebas - Resumen}
Para comenzar el proyecto en Flask se necesitó tener instalado Visual Studio Code desde donde programaríamos nuestra web app, para ello hay que realizar unas instalaciones de requisitos que iban a ser necesarias para el desarrollo del proyecto desde el VSCode:
\begin{enumerate}
    \item Creación de un entorno virtual, para aislar las dependencias que requiere el proyecto de las ya existentes.
    \item Pylint como verificador del código que escribimos.
    \item Git Tortoise para facilitar las subidas a GitHub.
    \item Instalar extensiones de VSCode.
    \item Registrar la aplicación en Azure Active Directory.
\end{enumerate}
Para comenzar el proyecto hay que hacer un primer registro de la aplicación en Azure Active Directory (explicado en el manual del programador). Esto permite acceder a la sesión de Outlook y poder tener un token de acceso para recoger información sobre esta sesión y comenzar a trabajar con los calendarios de cada uno.\newline


Otro aspecto a remarcar es la conexión con la base de datos, al subir todo a los servicios de Azure, la conexión ya no se puede realizar como al principio, que se hacía con la base de datos local de MySQL, ahora se realiza una conexión de pyodbc especificando el servidor, la base de datos, el usuario, contraseña y driver a utilizar desde Azure. Esto no permitirá que sea ejecutado en cualquier servidor local como se verá más adelante en la explicación del despliegue en el manual del programador. Se tuvo que modificar las consultas que realizaba ya que se tuvo que crear una nueva conexión distinta a la ya existente y el lenguaje de consulta cambia de MySQL al utilizarlo en SQL Database proporcionado por Azure.\newline



\subsection{Adaptación y dificultades en el proceso de desarrollo}
En esta sección nos centraremos exclusivamente en los problemas que fueron surgiendo durante el desarrollo y como los tratamos.\newline
Más adelante vimos que nos resultaría más cómodo trabajar mediante el uso de calendarios compartidos ya que todos serían gestionados desde una cuenta administrador en Outlook y estos pueden ser compartidos con los permisos deseados a los propietarios de las aulas, que serán los encargados de realizar reservas sobre dichos calendarios. De esta manera generamos calendarios globales, que equivalen a aulas, y están disponibles para todos los usuarios con los que se compartan. El problema de esto se ve a continuación, ya que compartir estos calendarios tiene que ser una función a realizar por el administrador manualmente.\newline

Como se acaba de mencionar, el error que no se ha podido solventar es automatizar el compartir calendarios desde la aplicación web, ya que no proporciona esta funcionalidad la API REST de Microsoft. Se estuvo buscando en la documentación de la que obtenemos todas las funcionalidades de dicha API REST y lo único que encontramos, fue una funcionalidad en fase Beta aún parte de ella aún y la funcionalidad que nos ofrece es la de gestionar los permisos de un calendario compartido con algún usuario, como vemos en la cita textual de esta página \textit{'Antes de que se pueda aplicar el uso compartido del calendario o la delegación, el propietario envía una invitación a una persona con la que compartir o delegado, y dicha persona o delegado acepta la invitación'}\cite{compartirCalendario}. Investigando y dando vueltas a este problema, se llegó a la conclusión de que la única forma de compartirlos es manualmente desde la cuenta del administrador. En el manual del programador se explica como compartir los calendarios a los propietarios.\newline 

Un aspecto relevante del proyecto fue la elección del punto de enlace de Microsoft Graph en vez del \textit{endpoint} de la API de Outlook, ya que ofrece más servicios y características, como por ejemplo los más usados en la aplicación que son usuarios, calendarios o correos, permitiendo que desde el mismo punto de conexión se trabaje con todo esto. Elección tomada según la documentación de Microsoft \cite{microsoftGraphCompare}. Primeramente se estuvo trabajando con el extremo de la API de Outlook (https://outlook.office.com/api) en la v2.0, ya que parecía más nuevo y posiblemente mejor, pero fueron apareciendo errores a la hora de utilizarlo y falta de características. Lo que provocó invertir varios días en una línea infructuosa que retrasó el avance del sprint, teniendo que buscar una solución, y finalmente encontrando que cambiar los puntos de conexión a Microsoft Graph era la mejor, esto significó cambiar la dirección URL a la que se realizan las llamadas de las APIs, adaptando los correspondientes parámetros según el tipo de llamada.\newline

Durante el desarrollo fueron surgiendo problemas y diferencias en el uso de las API con el tutorial de Microsoft, ya que está diseñado para Django y no para Flask, aquí se tuvo que decidir si cambiar el comienzo del proyecto y empezar de nuevo con Django para tener este aspecto solucionado o buscar la alternativa en el lenguaje de Flask, ya que la imposición del tutor para el proyecto fue utilizar este framework en el desarrollo, por lo que la solución tomada fue adaptar el código, esto implicó cambios en el código disponible en la documentación existente en \cite{pythonMicrosoftGraph} y obligó a buscar una adaptación de distintos características que tenían que ser usadas. Por ejemplo, para obtener el JWT (JSON Web Token), no sirve la función proporcionada ya que se necesitan diferentes argumentos en la función, mediante un identificador de token  proporcionado por una llamada a la API REST en este caso.\newline

\subsection{Despliegue en Azure}
En este apartado se aborda otro aspecto importante, que fue la decisión de subir el proyecto a Azure, ya que se había registrado la aplicación en Azure Active Directory y facilitaba una base de datos y permitía alojar la aplicación web sin coste.\newline
En cuanto al trabajo con Azure, comentar que para realizar la subida se tuvo que reestructurar el proyecto según lo que Microsoft establece al trabajar en un proyecto que utiliza VSCode y Flask \cite{pythonSample} \cite{flaskTutorialVSCode}.
Esto conllevó errores en las partes del código que se importaban al cambiar la carpeta raíz desde VSCode. Siguiendo las mismas especificaciones también generé el fichero \textit{launch} para poder ejecutar desde local mediante el debug.\newline
Una vez que el proyecto estaba estructurado se siguieron los pasos del tutorial que proporciona Miscrosoft para realizar la subida de la aplicación web a los servidores de Azure. \href{https://docs.microsoft.com/es-es/azure/developer/python/tutorial-deploy-app-service-on-linux-01}{Despliegue en Azure}\cite{deployVSCode}.\newline

Por último quiero comentar el problema que se tuvo con la suscripción de Azure. En primera instancia se disponía de una suscripción gratuita de estudiante, desde la cuál no se consiguió mantener abiertos los servicios desplegados las 24 horas del días, para poder ejecutarlo hay que iniciar el app service y la base de datos previamente. Pero al final del proyecto se agotó la prueba gratuita de estudiante desde la que ejecutaba el app service y la base de datos, por lo que se tuvo que contactar con el soporte de Azure y pedir un aumento del tipo de suscripción, ya que no se podía realizar esto manualmente, de modo que convirtieron la suscripción de estudiante en una suscripción por pago, lo que hace posible tenerlo subido el día de la defensa sin tener que iniciarlo justo antes manualmente.

\subsection{Aportaciones}
En esta sección se van a tratar las aportaciones a nivel personal más interesantes del proyecto.\newline
En primera instancia, me gustaría comentar las consultas que se realizan a la base de datos, tanto a la hora de reservar un aula para comprobar si está libre como a la hora de comprobar los eventos de un aula en concreto, ya que estas acciones no las realiza la API REST. Esto se realiza mediante unas SELECT que comprueban todos los posibles solapamientos dividiendo las fechas en el día y las horas para poder comparar con más precisión. A la hora de consultar las reservas de las aulas, dependiendo de los filtros que se establezcan se realizará una consulta u otra, lo mismo para a la hora de realizar la reserva.\newline
También hemos abordado el tema de las reservas en un entorno multiusuario, de forma que al intentar realizar una reserva, se ejecuta una primera consulta sobre el aula en el que se va a realizar la reserva, de tipo \textit{HOLDLOCK} provocando el bloqueo de esta para cualquier otro usuario hasta que termine la transacción.

