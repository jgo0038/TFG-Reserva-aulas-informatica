\apendice{Especificación de Requisitos}

\section{Introducción}
En este apartado de la documentación explicaré  los objetivos que se perseguían al comienzo del proyecto, los requisitos funcionales y no funcionales.
\section{Objetivos generales}
\begin{itemize}
	\item Desarrollar una aplicación web cliente servidor, en la que los usuarios con permisos necesarios puedan reservas aulas para los profesores.
	\item Ofrecer la posibilidad de reservar un único día, más de uno seguidos o un día de la semana durante un periodo de tiempo, un mismo aula.
	\item Ofrecer al usuario una vista sobre las aulas existentes, si es administrador podrá modificarlas o borrarlas también.
	\item Ofrecer al usuario una vista con las franjas horarias reservadas y libres de cada aula.
	\item Controlar el solapamiento de horas.
	\item Ofrecer un control al administrador sobre las reservas y la información de las aulas.
	\item Controlar el acceso de usuarios mediante el inicio de sesión de Microsoft.
	\item Que las reservas queden registradas en el calendario de Microsoft.
	\item Que las reservas queden sincronizadas con la base de datos.
	\item Que las altas bajas y modificaciones de reservas sean rastreables por el administrador mediante un sistema de auditoría.
\end{itemize}
\section{Catalogo de requisitos}
En esta sección enumeraré los requisitos funcionales de la aplicación y los usuarios que intervienen.\newline
\textbf{Propietario:} Centro o Departamento universitario que tiene asignado el control y mantenimiento de una serie de aulas.
\textbf{Usuario:} será quién use la aplicación, según los privilegios les hay de tres tipos:
    \begin{itemize}
        \item Sin privilegios / sin registrar: Puede consultar todas las aulas y su ocupación.
        \item Responsable: Persona designada por un propietario para que pueda realizar, modificar y borrar reservas sobre las aulas asignadas al mismo.
        \item Administrador: Puede consultar todas las aulas y su ocupación. Además puede realizar reservas sobre todas las aulas. Puede modificar la información de las aulas o borrarlas. Se encarga de asignar aulas en propiedad a los propietarios.
    \end{itemize}
\subsection{Requisitos funcionales} 
\begin{itemize}
    \item \textbf{RF.1} Controlar el acceso mediante un inicio de sesión.
    \begin{itemize}
        \item \textbf{RF.1.1} Acceso a la aplicación mediante el login de Outlook.
    \end{itemize}
    \item \textbf{RF.2} La aplicación tiene que poder distinguir entre el tipo de usuario que ha accedido. Distingue cuatro tipos:
    \begin{itemize}
        \item \textbf{RF.2.1} Diferencia los usuarios sin permisos.
        \item \textbf{RF.2.2} Diferencia los usuarios responsables de aulas.
        \item \textbf{RF.2.3} Diferencia los usuarios administradores.
        \item \textbf{RF.2.4} Diferencia los usuarios invitados que no inician sesión.
    \end{itemize}
    \item \textbf{RF.3} La aplicación tiene que poder mostrar información sobre las aulas.
    \begin{itemize}
        \item \textbf{RF.3.1} Permite ver el listado de aulas.
        \item \textbf{RF.3.2} Permite ver características de cada aula.
    \end{itemize}
    \item \textbf{RF.4} La aplicación tiene que poder modificar las aulas.
    \begin{itemize}
        \item \textbf{RF.4.1} Permite modificar las características de las aulas.
        \item \textbf{RF.4.2} Permite el borrado de aulas.
    \end{itemize}
    \item \textbf{RF.5} La aplicación tiene que poder realizar una reserva de tres tipos y reflejar de forma sincronizada esas reservas en el calendario de Outlook.
    \begin{itemize}
        \item \textbf{RF.5.1} Permite reservar un sólo día de una hora a otra especificada.
        \item \textbf{RF.5.2} Permite reservar varios seguidos a la misma hora.
        \item \textbf{RF.5.3} Permite reservar un día a la semana a la misma hora durante el tiempo especificado. 
    \end{itemize}
    \item \textbf{RF.6} La aplicación tiene que poder mostrar todos los eventos creados.
    \begin{itemize}
        \item \textbf{RF.6.1} Permite modificar los eventos y reflejar de forma sincronizada esas reservas en el calendario de Outlook.
        \item \textbf{RF.6.2} Permite borrar los eventos y eliminar de forma sincronizada esas reservas en el calendario de Outlook.
    \end{itemize}
    \item \textbf{RF.7} La aplicación tiene que poder crear un aula nueva para un manejo posterior.
    \item \textbf{RF.8} La aplicación tiene que permitir al administrador ver  una tabla de auditoría con las altas, bajas y modificaciones que se realizan en las reservas.
    \item \textbf{RF.9} La aplicación tiene que poder crear y modificar los propietarios y su información.
    
\end{itemize}

\subsection{Requisitos no funcionales}
\begin{itemize}
	\item \textbf{RNF.1 Usabilidad} Proporcionar una página web sencilla de usar e intuitiva.
	\item \textbf{RNF.2 Disponibilidad} El tiempo para iniciar o reiniciar el sistema no es mayor a 5 minutos.
	\item \textbf{RNF.3 Seguridad} Se controla que no todos los usuarios puedan acceder sin haber iniciado sesión a realizar modificaciones y una vez iniciada sesión, se controla el acceso a las distintas partes de la aplicación según los privilegios.
	\item \textbf{RNF.4 Mantenibilidad} Es sencillo añadir nuevas funcionalidades.
	\item \textbf{RNF.5 Sincronización con Outlook} Las reservas de las aulas tendrán que estar sincronizadas en la base de datos como en el calendario de Outlook.
\end{itemize}
\newpage
\section{Especificación de requisitos}
\subsection{Diagrama de casos de uso}

\imagen{CU_SinPermisos}{Diagrama de casos de uso de un usuario sin permisos.}
\imagen{CU_Responsable}{Diagrama de casos de uso de un usuario responsable.}
\imagen{CU_Administrador}{Diagrama de casos de uso de un usuario administrador.}

\subsection{Especificación de los casos de uso}
\tablaSmallSinColores{Caso de uso 0: Iniciar sesión.}{p{3cm} p{.75cm} p{9.5cm}}{tablaUC0}{
  \multicolumn{3}{l}{Caso de uso 0: Iniciar sesión.} \\
 }
 {
  Descripción                            & \multicolumn{2}{p{10.25cm}}{Permite al usuario iniciar sesión en la aplicación.} \\\hline
  \multirow{6}{3.5cm}{Requisitos}   &\multicolumn{2}{p{10.25cm}}{RF-1} \\\cline{2-3}
                                         & \multicolumn{2}{p{10.25cm}}{RF-1.1} 
                                         \\\cline{2-3}
                                         & \multicolumn{2}{p{10.25cm}}{RF-2}
                                         \\\cline{2-3}
                                         &
                                    \multicolumn{2}{p{10.25cm}}{RF-2.1}
                                    \\\cline{2-3}
                                         &
                                    \multicolumn{2}{p{10.25cm}}{RF-2.2}
                                    \\\cline{2-3}
                                         &
                                    \multicolumn{2}{p{10.25cm}}{RF-2.3}
                                         \\\hline
  Precondiciones                         &  \multicolumn{2}{p{10.25cm}}{Disponer de una cuenta de Outlook.}   \\\hline
  \multirow{2}{3.5cm}{Secuencia normal}  & Paso & Acción \\\cline{2-3}
                                         & 1    & El usuario pincha el botón 'Iniciar sesión'.
  \\\cline{2-3}
                                         & 2    & Se redirige al inicio de sesión de Outlook.
  \\\cline{2-3}
                                         & 3    & Se introducen los datos de inicio de sesión de la cuenta de Outlook.
    \\\cline{2-3}
                                         & 4    & Se inicia la sesión si existe la cuenta.
                                         \\\hline
  Postcondiciones                        & \multicolumn{2}{p{10.25cm}}{Si la sesión se ha iniciado correctamente, se lleva al usuario a la página de inicio. Si no se vuelve a pedir que introduzca una cuenta existente.} \\\hline
  Excepciones                        & \multicolumn{2}{p{10.25cm}}{Cuenta de Outlook inexistente.}
\\\hline
}



\tablaSmallSinColores{Caso de uso 1: Consultar eventos de aulas.}{p{3cm} p{.75cm} p{9.5cm}}{tablaUC1}{
  \multicolumn{3}{l}{Caso de uso 1: Consultar eventos de aulas.} \\
 }
 {
  Descripción                            & \multicolumn{2}{p{10.25cm}}{Permite al usuario visualizar los eventos creados en las aulas, facilitando filtros para una búsqueda mas concreta. Permite consultar la disponibilidad de aulas antes de llevar a cabo una petición de reserva al responsable.} \\\hline
  \multirow{1}{3.5cm}{Requisitos}   &\multicolumn{2}{p{10.25cm}}{RF-6} 
                                         \\\hline

    \multirow{2}{3.5cm}{Secuencia normal}  & Paso & Acción \\\cline{2-3}
                                         & 1    & El usuario pincha el botón 'Consultar ocupación de aulas' de la barra superior de navegación o el menú principal.
  \\\cline{2-3}
                                         & 2    & Se redirige a una página donde se muestra un desplegable para elegir el edificio. Con un botón de 'Mostrar aulas'.
  \\\cline{2-3}
                                         & 3    & Se elige el edifico y se pincha en 'Mostrar aulas', aparece entonces un desplegable con las aulas de éste y una primera opción 'Cualquiera'. Debajo tenemos unos cuantos campos para filtrar las aulas por características si elegimos la opción 'Cualquiera' anteriormente.
                                         
  \\\cline{2-3}                                         
                                    & 4     & Si elegimos un aula en concreto solamente se tendrán en cuenta los campos de fecha y hora obligatorios. 
                                    
  \\\cline{2-3}                                 
                                    & 5     &  Si no se especifica capacidad ni ordenadores, se mostrarán de todas las capacidades y ordenadores, si se especifican se mostrarán eventos con igual o mayor número.                                  
    \\\cline{2-3}
                                         & 6    & Una vez se ha rellenado el formulario al menos con los campos de fecha y horas se puede pinchar en el botón 'Enviar consulta' para consultar los eventos que coincidan con los especificado anteriormente.
     \\\cline{2-3}
                                        & 7     & Se muestra la tabla con los eventos futuros que cumplan las especificaciones, mostrando detalles sobre ellos. Existe un botón en la parte inferior de 'Mostrar eventos pasados' que mostrará los eventos pasados coincidentes en color rojo si en la fecha de inicio se puso una anterior a la actual. Una vez mostrados se puede volver a ocultar del botón 'Ocultar eventos pasados'.
                                        \\\hline

                                        
  Postcondiciones                        & \multicolumn{2}{p{10.25cm}}{Permite la visualización de los eventos según el aula elegido, depende de los privilegios del usuario dispondrá de las opciones de modificar (CU1.1) y borrar estos eventos (CU1.2).} \\\hline
  Excepciones                        & \multicolumn{2}{p{10.25cm}}{Elegir un aula en concreto en el paso 3. En ese caso ejecutar solo pasos 4, 6 y 7}
\\\hline
}


\tablaSmallSinColores{Caso de uso 1.1: Modificar eventos.}{p{3cm} p{.75cm} p{9.5cm}}{tablaUC1.1}{
  \multicolumn{3}{l}{Caso de uso 1.1: Modificar eventos.} \\
 }
 {
  Descripción                            & \multicolumn{2}{p{10.25cm}}{Permite al usuario modificar los eventos existentes.} \\\hline
  \multirow{1}{3.5cm}{Requisitos}   &\multicolumn{2}{p{10.25cm}}{RF-6} 
  \\\cline{2-3}
        &       \multicolumn{2}{p{10.25cm}}{RF-6.1}
                                         \\\hline
  Precondiciones                         &  \multicolumn{2}{p{10.25cm}}{Haber iniciado sesión (CU0) como administrador o propietario, y haber ejecutado el una consulta de eventos de aulas (CU1)}   \\\hline
    \multirow{2}{3.5cm}{Secuencia normal}  & Paso & Acción \\\cline{2-3}

                                        & 1     & Se muestra la tabla con los eventos, especificando detalles sobre ellos.
    \\\cline{2-3}
                                        & 2     & Se ve un botón de 'Modificar' al lado de los eventos sobre los que tenga permiso para modificar.
    \\\cline{2-3}
                                        & 3     & Se pincha en el botón 'Modificar' y te abre un formulario en la parte superior para cambiar los datos del evento.
    \\\cline{2-3}
                                        & 4     & Una vez modificados los datos, se pincha sobre el botón 'Guardar' y se realizan las modificaciones.   
    \\\cline{2-3}
                                        & 5     &              Si se modifica correctamente, se vuelve a la página anterior.                   
                                        \\\hline

                                        
  Postcondiciones                        & \multicolumn{2}{p{10.25cm}}{El evento ha sido modificado, y la fecha, horas y aula tras la modificación son compatibles con el resto de eventos. Además la modificación está registrada de forma sincronizada en la BD y en Outlook.} \\\hline
  Excepciones                        & \multicolumn{2}{p{10.25cm}}{Cambiar a una fecha ya reservada.}
\\\hline
}




\tablaSmallSinColores{Caso de uso 1.2: Borrar eventos.}{p{3cm} p{.75cm} p{9.5cm}}{tablaUC1.2}{
  \multicolumn{3}{l}{Caso de uso 1.2: Borrar eventos.} \\
 }
 {
  Descripción                            & \multicolumn{2}{p{10.25cm}}{Permite al usuario borrar los eventos existentes.} \\\hline
  \multirow{1}{3.5cm}{Requisitos}   &\multicolumn{2}{p{10.25cm}}{RF-6} 
  \\\cline{2-3}
        &       \multicolumn{2}{p{10.25cm}}{RF-6.2}
                                         \\\hline
  Precondiciones                         &  \multicolumn{2}{p{10.25cm}}{Haber iniciado sesión (CU0) como administrador o propietario, y haber ejecutado el una consulta de eventos de aulas (CU1)}   \\\hline
    \multirow{2}{3.5cm}{Secuencia normal}  & Paso & Acción \\\cline{2-3}
                                        & 1     & Se muestra la tabla con los eventos, especificando detalles sobre ellos.
    \\\cline{2-3}
                                        & 2     & Se ve un botón de 'Borrar' al lado de los eventos sobre los que tenga permiso para borrar.
    \\\cline{2-3}
                                        & 3     & Se pincha en el botón 'Borrar' y te pide la confirmación del usuario para borrar el evento.
                                        \\\hline

                                        
  Postcondiciones                        & \multicolumn{2}{p{10.25cm}}{Si se borra correctamente, se muestra un mensaje por pantalla y se vuelve a la página anterior. Si se cancela el borrado te mantienes en la misma.} \\\hline

\\\hline
}


\tablaSmallSinColores{Caso de uso 2: Consultar información sobre las aulas.}{p{3cm} p{.75cm} p{9.5cm}}{tablaUC2}{
  \multicolumn{3}{l}{Caso de uso 2: Consultar información sobre las aulas.} \\
 }
 {
  Descripción                            & \multicolumn{2}{p{10.25cm}}{Permite al usuario conocer la información relevante acerca de las aulas existentes, sea o no responsable de ellas. La información que se facilita es: nombre, edificio, tipo de aula, capacidad, nº de ordenadores y propietario.} \\\hline
  \multirow{1}{3.5cm}{Requisitos}   &\multicolumn{2}{p{10.25cm}}{RF-3}
                                    \\\cline{2-3}
                                    & \multicolumn{2}{p{10.25cm}}{RF-3.1} \\\cline{2-3}
                                    & \multicolumn{2}{p{10.25cm}}{RF-3.2}
                                         \\\hline
  Precondiciones                         &  \multicolumn{2}{p{10.25cm}}{Haber iniciado sesión.}   \\\hline
    \multirow{2}{3.5cm}{Secuencia normal}  & Paso & Acción \\\cline{2-3}
                                         & 1    & El usuario pincha el botón 'Ver aulas' de la barra superior de navegación.
  \\\cline{2-3}
                                         & 2    & Se redirige a la página de información sobre las aulas, donde se muestra una opción para elegir el grupo de aulas de entre todos los existentes.
  \\\cline{2-3}
                                         & 3    & Una vez se elige el grupo de aulas, se pincha en el botón 'Ver aulas'.
    \\\cline{2-3}
                                         & 4    & Se muestra una tabla con las aulas y su información del edificio elegido en el paso anterior.

                                        \\\hline

                                        
  Postcondiciones                        & \multicolumn{2}{p{10.25cm}}{Permite la visualización de las características de las aulas, depende de los privilegios del usuario dispondrá de las opciones de modificar (CU2.1) y borrar (CU2.2) estas aulas.} \\\hline
\\\hline
}


\tablaSmallSinColores{Caso de uso 2.1: Modificar aulas.}{p{3cm} p{.75cm} p{9.5cm}}{tablaUC2.1}{
  \multicolumn{3}{l}{Caso de uso 2.1: Modificar aulas.} \\
 }
 {
  Descripción                            & \multicolumn{2}{p{10.25cm}}{Permite al usuario modificar la información relevante acerca de las aulas existentes. } \\\hline
  \multirow{1}{3.5cm}{Requisitos}   &\multicolumn{2}{p{10.25cm}}{RF-4}
                                    \\\cline{2-3}
                                    & \multicolumn{2}{p{10.25cm}}{RF-4.1} 
                                         \\\hline
  Precondiciones                         &  \multicolumn{2}{p{10.25cm}}{Haber iniciado sesión (CU0) como administrador o propietario, y haber ejecutado el una consulta de aulas (CU1)}   \\\hline
    \multirow{2}{3.5cm}{Secuencia normal}  & Paso & Acción \\\cline{2-3}
                                         & 1    & Se ve un botón de 'Editar' al lado de la información de las aulas que pueda modificar.
   \\\cline{2-3}
                                         & 2    & Al pinchar sobre éste botón se abre un cuadro en la parte superior de la pantalla donde se pueden introducir los valores a modificar.
   \\\cline{2-3}
                                         & 3    & Una vez se han modificado los datos correspondientes se pincha en el botón 'Guardar' y se actualizan los datos.      
   \\\cline{2-3}
                                         & 4    & Si se cambia el propietario, se mostrará un mensaje recordando el cambio manual de propietario que tiene que hacer el administrador.   
   \\\cline{2-3}
                                         & 5    & Si se cambia correctamente, se vuelve a la pantalla anterior y se muestra un mensaje por pantalla.                                           

                                        \\\hline

                                        
  Postcondiciones                        & \multicolumn{2}{p{10.25cm}}{El cambio en el aula han sido registrado en la base de datos.}
  \\\cline{2-3} &   \multicolumn{2}{p{10.25cm}}{El cambio de propietario en Outlook queda pendiente de hacer manualmente.}
  
  \\\hline
  Excepciones                        & \multicolumn{2}{p{10.25cm}}{Pinchar en el botón de 'Cancelar', con lo que se cancelan los cambios.}
  \\\cline{2-3} &  \multicolumn{2}{p{10.25cm}}{Dejar sin rellenar partes del formulario de modificación.}
\\\hline
}


\tablaSmallSinColores{Caso de uso 2.2: Borrar aulas.}{p{3cm} p{.75cm} p{9.5cm}}{tablaUC2.2}{
  \multicolumn{3}{l}{Caso de uso 2.2: Borrar aulas.} \\
 }
 {
  Descripción                            & \multicolumn{2}{p{10.25cm}}{Permite al usuario borrar aulas existentes. } \\\hline
  \multirow{1}{3.5cm}{Requisitos}   &\multicolumn{2}{p{10.25cm}}{RF-4.2}
                                    
                                         \\\hline
  Precondiciones                         &  \multicolumn{2}{p{10.25cm}}{Haber iniciado sesión (CU0) como administrador o propietario, y haber ejecutado el una consulta de aulas (CU1)}   \\\hline
    \multirow{2}{3.5cm}{Secuencia normal}  & Paso & Acción \\\cline{2-3}
                                         & 1    & Se muestra una tabla con las aulas y su información del edificio elegido en el paso anterior.
    \\\cline{2-3}
                                         & 2    & Se ve un botón de 'Borrar' al lado de la información de las aulas que pueda eliminar.
                                        
   \\\cline{2-3}
                                         & 3    & Al pinchar sobre éste botón se abre un cuadro pidiendo confirmación del usuario para borrar el aula. Ofreciendo una comprobación en la que se permite borrar definitivamente o cancelar el borrado.
\\\cline{2-3}
                                         & 4    & Si se borra el aula,se muestra un mensaje por pantalla avisando del borrado y se vuelve al formulario anterior.                                         

                                        \\\hline

                                        
  Postcondiciones                        & \multicolumn{2}{p{10.25cm}}{El borrado del aula ha sido registrado en la base de datos y en el calendario de Outlook.}
  
  \\\hline
  Excepciones                        & \multicolumn{2}{p{10.25cm}}{Pinchar en el botón de cancelar a la hora de la confirmación.}
\\\hline
}



\tablaSmallSinColores{Caso de uso 3: Realizar reservas.}{p{3cm} p{.75cm} p{9.5cm}}{tablaUC3}{
  \multicolumn{3}{l}{Caso de uso 3: Realizar reservas.} \\
 }
 {
  Descripción                            & \multicolumn{2}{p{10.25cm}}{Permite al usuario realizar reservas sobre las aulas que tenga permisos siempre que la hora no esté ocupada.} \\\hline
  \multirow{1}{3.5cm}{Requisitos}   &\multicolumn{2}{p{10.25cm}}{RF-5}
                                    \\\cline{2-3}
                                    & \multicolumn{2}{p{10.25cm}}{RF-5.1} \\\cline{2-3}
                                    & \multicolumn{2}{p{10.25cm}}{RF-5.2}
                                    \\\cline{2-3}
                                    & \multicolumn{2}{p{10.25cm}}{RF-5.3}
                                         \\\hline
  Precondiciones                         &  \multicolumn{2}{p{10.25cm}}{Haberse logueado como administrador o responsable (CU0).}   \\\hline
    \multirow{2}{3.5cm}{Secuencia normal}  & Paso & Acción \\\cline{2-3}
                                         & 1    & El usuario (administrador o propietario) pincha el botón 'Realizar reserva' de la barra superior de navegación o el menú principal.
  \\\cline{2-3}
                                         & 2    & Se muestra un formulario en el que se aprecian 3 divisiones. En el primero se piden la capacidad, nº de ordenadores, edificio y tipo de aula.
  \\\cline{2-3}
                                         & 3    & Hay que pinchar en el botón 'Consultar' para continuar y se llena el campo 'seleccionar aula' con las opciones según lo especificado al comienzo, si hay más de un aula se añade una opción de 'Cualquiera' a la lista de aulas disponibles.
\\\cline{2-3}
                                         & 4    & En la segunda división del cuestionario, se ofrece la posibilidad de cambiar el tipo de reserva. La opción de pinchar en el botón 'Aplicar' conlleva que cambie el número de casillas a rellenar dependiendo de la opción que se haya elegido.
\\\cline{2-3}
                                        & 5    & Si se elige 'un día' se queda igual. Si se elige la opción 'días contiguos' se añaden el campo 'Fecha fin'. Si se elige la opción 'Un día por semana' se añaden los campos 'Selecciona día de la semana' y 'Fecha fin'
\\\cline{2-3}
                                         & 6    & Si se pincha en el botón 'Consultar ocupación de aulas', se muestran las horas reservadas del aula elegida, si la elección es 'Cualquiera', se muestran los eventos de todas las aulas del desplegable.
\\\cline{2-3}
                                         & 7    & Si se pincha en el botón 'Reservar' se realiza la reserva la hora y aula pedida si están libres.
\\\cline{2-3}
                                         & 8    & Se permite volver a reservar otro aula recordando la reserva recién realizada por si se quiere una mínima modificación.                                 

\\\cline{2-3}
                                         & 9    & Si se realiza correctamente la reserva, se envía un email al profesor que se le ha indicado en el formulario avisando de la reserva realizada.

                                        \\\hline    

                                        
  Postcondiciones                        & \multicolumn{2}{p{10.25cm}}{La reserva ha sido registrada en la base de datos y en Outlook.} \\\hline
  Excepciones                        & \multicolumn{2}{p{10.25cm}}{No existan aulas con los requisitos específicos, con lo que se muestra un mensaje por pantalla indicándolo.}
\\\hline
}




\tablaSmallSinColores{Caso de uso 4: Crear aulas.}{p{3cm} p{.75cm} p{9.5cm}}{tablaUC4}{
  \multicolumn{3}{l}{Caso de uso 4: Crear aulas.} \\
 }
 {
  Descripción                            & \multicolumn{2}{p{10.25cm}}{Permite al usuario administrador crear aulas nuevas.} \\\hline
  \multirow{1}{3.5cm}{Requisitos}   &\multicolumn{2}{p{10.25cm}}{RF-7}
                                    \\\cline{2-3}
                                         \\\hline
Precondiciones                         &  \multicolumn{2}{p{10.25cm}}{Haberse logueado como un usuario administrador.}   \\\hline
    \multirow{2}{3.5cm}{Secuencia normal}  & Paso & Acción \\\cline{2-3}
                                         & 1    & El usuario (administrador) pincha el botón 'Añadir aulas' de la barra superior de navegación.
  \\\cline{2-3}
                                         & 2    & Se muestra un formulario en el que se puede seleccionar el edificio donde se ubica el aula, el tipo de aula y el propietario. Se dejan campos a rellenar para proporcionar el nombre del aula, la capacidad y el nº de ordenadores.
  \\\cline{2-3}
                                         & 3    & Se pincha en el botón 'Crear aula'.
  \\\cline{2-3}
                                         & 4    & Asignar manualmente los permisos desde Outlook al propietario del nuevo aula.                                       


                                        \\\hline
 Postcondiciones                        & \multicolumn{2}{p{10.25cm}}{El aula está creada en la base de datos y en el calendario de Outlook, pero pendiente de asignarla permisos a sus responsables desde Oulook.} \\\hline
  Excepciones                        & \multicolumn{2}{p{10.25cm}}
    {No llenar todos los campos requeridos}
                        \\\hline

}
                                        
                                        
\tablaSmallSinColores{Caso de uso 5: Ver auditorías.}{p{3cm} p{.75cm} p{9.5cm}}{tablaUC5}{
  \multicolumn{3}{l}{Caso de uso 5: Ver auditorías.} \\
 }
 {
  Descripción                            & \multicolumn{2}{p{10.25cm}}{Permite al usuario administrador ver las altas, bajas y modificaciones realizadas sobre cualquier reserva.} \\\hline
  \multirow{1}{3.5cm}{Requisitos}   &\multicolumn{2}{p{10.25cm}}{RF-8}
                                         \\\hline
Precondiciones                         &  \multicolumn{2}{p{10.25cm}}{Haberse logueado como un usuario administrador.}   \\\hline
    \multirow{2}{3.5cm}{Secuencia normal}  & Paso & Acción \\\cline{2-3}
                                         & 1    & El usuario (administrador) pincha el botón 'Ver auditorías' de la barra superior de navegación.
  \\\cline{2-3}
                                         & 2    & Se muestra una tabla en la que figuran las acciones sobre las reservas, ya sea una alta, baja o modificación. Se especifica el usuario y la fecha de dicha acción.


                                        \\\hline

}       




\tablaSmallSinColores{Caso de uso 6: Ver/editar propietarios.}{p{3cm} p{.75cm} p{9.5cm}}{tablaUC5}{
  \multicolumn{3}{l}{Caso de uso 6: Crear/editar propietarios.} \\
 }
 {
  Descripción                            & \multicolumn{2}{p{10.25cm}}{Permite al usuario administrador ver y modificar la información de los propietarios existentes.} \\\hline
  \multirow{1}{3.5cm}{Requisitos}   &\multicolumn{2}{p{10.25cm}}{RF-9}
                                         \\\hline
Precondiciones                         &  \multicolumn{2}{p{10.25cm}}{Haberse logueado como un usuario administrador.}   \\\hline
    \multirow{2}{3.5cm}{Secuencia normal}  & Paso & Acción \\\cline{2-3}
                                         & 1    & El usuario (administrador) pincha el botón 'Ver/Editar propietarios' de la barra superior de navegación.
  \\\cline{2-3}
                                         & 2    & Se muestra una tabla en la que figuran los propietarios existentes con su información relevante, incluyendo el responsable.
  \\\cline{2-3}
                                         & 3    & Se da la opción de modificar la información de cada propietario pinchando en el botón 'Modificar' al lado de cada uno.
  \\\cline{2-3}
                                         & 4    & Se muestra un formulario en el que se permite editar los campos del propietario a modificar .     
  \\\cline{2-3}
                                         & 5    & Se visualizan los cambios realizados.                                  

                                        \\\hline
 Postcondiciones                        & \multicolumn{2}{p{10.25cm}}{Se actualiza la BD con el nuevo propietario y/o las modificaciones de sus datos.} \\\hline

                        \\\hline

}  


\tablaSmallSinColores{Caso de uso 6.1: Crear propietario.}{p{3cm} p{.75cm} p{9.5cm}}{tablaUC5}{
  \multicolumn{3}{l}{Caso de uso 6.1: Crear propietario.} \\
 }
 {
  Descripción                            & \multicolumn{2}{p{10.25cm}}{Permite al usuario administrador crear un nuevo propietario.} \\\hline
  \multirow{1}{3.5cm}{Requisitos}   &\multicolumn{2}{p{10.25cm}}{RF-9}
                                         \\\hline
Precondiciones                         &  \multicolumn{2}{p{10.25cm}}{Haber ejecutado el CU6 (Ver/editar propietarios).}   \\\hline
    \multirow{2}{3.5cm}{Secuencia normal}  & Paso & Acción \\\cline{2-3}
  \\\cline{2-3}
                                         & 1    & Se pincha en la opción 'Crear' y el propietario está disponible para ser creado.
  \\\cline{2-3}
                                         & 2    & Se introducen los datos que se quieran.
  \\\cline{2-3}
                                         & 3    & Se pincha en el botón de 'Crear' para confirmar los cambios.   
  \\\cline{2-3}
                                         & 4    & Se visualizan los cambios realizados.       
                                        \\\hline
 Postcondiciones                        & \multicolumn{2}{p{10.25cm}}{El nuevo propietario es insertado en la base de datos.}\\\hline
  Excepciones                        & \multicolumn{2}{p{10.25cm}}{Si se omite algún campo, no se permite la creación y se indica el campo en cuestión necesario.}

                        \\\hline

}  