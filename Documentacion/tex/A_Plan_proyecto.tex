\apendice{Plan de Proyecto Software}

\section{Introducción}
Un plan de proyecto recoge la planificación temporal donde se resumen las divisiones de trabajo en periodos de tiempo y un estudio de viabilidad teniendo en cuenta las repercusiones tanto legales como económicas de dicho proyecto.

\section{Planificación temporal}
Para la planificación temporal se ha usado la plataforma GitHub en la que se ha intentado simular una metodología \textit{Scrum}, en la que las divisiones de tiempo para realizar una determinada función se conocen como \textit{Sprints}. La duración de los Sprints en este caso no era especificada, ya que no se tenían unos plazos obligatorios y algunos sprints duraban más de lo planeado y otros menos. Los primeros Sprints están resumidos en unas \textit{issues} de mayor tamaño ya que se centraban en realizar pruebas y generación del comienzo del proyecto.\newline
Dentro de la plataforma de GitHub se utilizó el tablero de Zen-hub para controlar mejor la inicialización y finalización de las \textit{issues}.\newline
El enlace GitHub del repositorio del presente trabajo puede encontrarse en:\newline \url{https://github.com/jgo0038/TFG-Reserva-aulas-informatica} \newline

\subsection{Sprint 0 - 24/02/2020 - 27/02/2020}
Primer sprint en el que se inicia el proyecto, realmente comienza antes el proyecto y la investigación pero hasta entonces no se crea el GitHub. Las tareas en este sprint estaban destinadas a la investigación y la preparación para el proyecto:
\begin{itemize}
    \item Elegir los programas a utilizar. 
	\item Crear un repositorio en GitHub.
    \item Generar el entorno virtual de trabajo.
    \item Registrar la aplicación en Azure Active Directory para obtener un identificador de la aplicación.
	\item Implementación una primera versión simple del login en nuestra web app.
	\item Empezar a manejar Overleaf para la documentación el LaTex.
	\item Investigar sobre el uso de la API de Outlook para Flask.
\end{itemize}
Una vez se registró la app en Azure, el objetivo principal era poder iniciar sesión en Outlook con los códigos recibidos, utilizando un repositorio de GitHub como \href{https://github.com/brysontyrrell/Office-365-Flask-App}{ayuda}\cite{loginO365}.
Resaltar que aquí creo la primera rama donde se subieron estos primeros pasos del proyecto.
\subsection{Sprint 1 - 27/02/2020 - 08/03/2020}
En este sprint solamente hubo especificada una tarea, que engloba trabajar con la API y empezar a enviar y recibir datos para conocer su funcionamiento. Ya conociendo la forma en la que iba a trabajar y los requisitos que había que conseguir, se comenzó a hacer pruebas sobre la API solucionando los primeros errores y consiguiendo los primeros avances.\newline
Una vez estaba iniciada la sesión de Outlook el objetivo de este Sprint era leer y modificar eventos del calendario de la sesión. Se sube esto a nueva nueva rama, que permite, a parte del inicio de sesión ya comentado en el Sprint 0, la funcionalidad de poder recoger eventos del calendario principal del usuario que ha iniciado sesión en la web app.

\subsection{Sprint 2 - 08/03/2020 - 05/04/2020}
Los objetivos principales en este sprint fueron los siguientes:
\begin{itemize}
    \item Investigar y comenzar a trabajar con los calendarios compartidos de Outlook.
    \item Comenzar a trabajar con la base de datos local de MySQL.
    \item Investigar sobre la seguridad de la aplicación.
\end{itemize}
En este Sprint, la tarea de investigar la seguridad era secundaria y no corría prisa de momento, por lo que se relegó a mirarlo más tarde cuando tuviera más funcionalidad y páginas creadas. La tarea principal era el trabajo con los calendarios compartidos, en concreto se buscaba poder crear eventos en calendarios compartidos, que son los que hacen la función de albergar los eventos de cada aula.

\subsection{Sprint 3 - 26/03/2020 - 15/04/2020}
Los objetivos que abordamos en este Sprint se centrar en añadir funcionalidades a la aplicación y estudiar el despliegue:
\begin{itemize}
    \item Controlar el solapamiento de horas de los eventos en los calendarios ya que no lo realiza Outlook.
    \item Añadir una página en la que se puedan ver los eventos que hay creados en un determinado calendario.
    \item Estudiar las posibilidades que hay para desplegar la web app. Se opta por utilizar Azure.
\end{itemize}
Este Sprint se fue compaginando con el anterior, ya que la tarea de investigar sobre la seguridad de la aplicación había quedado pendiente.
\subsection{Sprint 4 - 05/04/2020 - 27/04/2020}
Este Sprint tenía como objetivos principales:
\begin{itemize}
    \item Controlar que funcione en un entorno multiusuario, que no se pueda reservar a la vez el mismo aula.
    \item Crear el primer modelo de las tablas y relaciones en la base de datos local.
    \item Comprobar que las fechas de inicio y fin coincidan en una reserva simple.
    \item Ofrecer al cliente una opción para elegir de qué calendario quiere ver las horas reservadas de entre aquellos de los que es responsable.
\end{itemize}
Este Sprint se da por cerrado pese a haber quedado una tarea abierta (sincronizar la web app con la base de datos, de modo que borre los eventos de la base de datos local y los vuelva a guardar recuperándolos directamente desde Outlook, como solución a algún problema o corrupción de los datos de la base de datos), esta tarea fue desestimada en una reunión posterior. La base de datos y los calendarios de Outlook siempre tienen que estar sincronizados, compartiendo los mismos datos en ambos sitios, de forma que si se borra un evento del calendario, también se borre el correspondiente en la base de datos. Si esto fallara, más adelante no se podrían realizar reservas en fechas que en realidad deberían de estar libres.
\subsection{Sprint 5 - 17/04/2020 - 07/05/2020}
Los objetivos que voy a abordar en este Sprint son los siguientes:
\begin{itemize}
    \item Realizar un cambio en una relación de la base de datos, permite albergar mas de un propietario a un aula.
    \item Crear el servidor de Azure database, se crea en lenguaje SQL y sin servidor, para reducir el coste al mínimo.
    \item Subir la base de datos local al servidor de Azure database creado y adaptar el lenguaje al SQL server.
    \item Conectar la base de datos subida a Azure con nuestra web app, se realiza mediante pyodbc.
    \item Organizar el proyecto según el estandar establecido por Microsoft y VSCode para poder subir facilmente a Azure.
    \item Desplegar la aplicación en Azure.
    \item Primera asignación y restricción de acceso a páginas según la cuenta que acceda.
\end{itemize}

\subsection{Sprint 6 - 07/05/2020 - 15/06/2020}
Este Sprint se centra en mejorar partes del código ya existentes y facilitar la reserva de las aulas, concretamente los objetivos fueron:
\begin{itemize}
    \item Solucionar los errores de los formularios, que no se muestran por pantalla correctamente.
    \item Añadir unos filtros antes de reservar el aula, limitando las aulas que se muestran. Añadir también un campo para el nombre del profesor que lo reserva.
    \item Crear una nueva página en la que el administrador pueda crear nuevas aulas.
    \item Crear una nueva página desde la que se pueda ver la información de cada aula.
    \item En la página de la información, permitir el modificar los datos de las aulas.
    \item Restringir los datos que se pueden introducir al modificar aulas en campos como propietario, edificio o tipo de aula.
    \item En la página de la información, permitir una opción para borrar las aulas.
    \item Añadir las opciones de reserva múltiple (reservar más de un día seguido a la misma hora) y de reserva periódica (reservar un día a la semana a la misma hora durante el tiempo que se quiera ).
    \item Crear una pagina de auditoría para el usuario administrador, desde la que puedan ver las altas, bajas y modificaciones que se realizan sobre los eventos.
\end{itemize}
Comentar que se actualiza también la base de datos, aplicando la actualización y borrado en cascada.
La fecha de finalización se alargó tanto ya que la tarea de la tabla de auditorías quedó pospuesta, de forma que si quedaba tiempo al final para realizarla se haría, por ello fue terminada el 15/06.

\subsection{Sprint 7 - 20/05/2020 - 17/06/2020}
El Sprint 7 se centra en volver a introducir el código con el que se ha estado trabajando en Azure e ir añadiendo la documentación al proyecto, para ello se fijan estos objetivos:
\begin{itemize}
    \item Actualizar el repositorio, añadiendo todo tal cual está con los cambios subidos a Azure.
    \item Añadir la carpeta de documentación siguiendo la plantilla proporcionada.
    \item Agregar la documentación realizada hasta entonces.
    \item Añadir los casos de uso.
\end{itemize}

\subsection{Sprint 8 - 26/05/2020 - 17/06/2020}
Este Sprint se centra en tareas más cortas y más visibles, dando estilos a la página y sus elementos y añadiendo funciones para mejorar el funcionamiento. Las tareas más destacables son las siguientes:
\begin{itemize}
    \item Añadir botón que permita ver las horas ocupadas del aula a reservar.
    \item Añadir cabeceras a las páginas.
    \item Añadir un campo nuevo de 'nº de ordenadores' a las aulas, si no tiene ordenadores se indicará con un 0. Establecer este campo como filtro al reservar aula también.
    \item Añadir campo de correo electrónico del nombre del profesor al que se reserva el aula y enviarle un mensaje cuando la reserva se realice.
    \item Representar los privilegios de cada usuario en los botones, de forma que se bloquee si no tiene permiso para esa función.
    \item Mostrar mensaje al modificar el campo propietario recordando al administrador que tiene que darle los permisos al nuevo propietario manualmente desde el calendario de Outlook.
    \item Aplicar estilos a los elementos de la página.
    \item Crear la barra de navegación.
    \item Implementar el cierre de sesión.
    \item Diseñar la página de inicio de la aplicación.
    \item Diseñar la página principal al iniciar sesión.
    \item Crear la barra de navegación en un fichero aparte del que puedan heredar dicho código.
\end{itemize}
Anótese que los filtros de búsqueda de capacidad y nº de ordenadores en la página de realizar reserva se atienen al mínimo buscado, no es un búsqueda exacta de esas características, sino se limitaría demasiado.

\subsection{Sprint 9 - 08/06/2020 - 26/06/2020}
Este sprint se crea después de una corrección completa del proyecto por el tutor. Anotando los fallos encontrados, las mejoras a realizar y funcionalidades pendientes:
\begin{itemize}
    \item Mejorar la página de eventos, dejando más claro su funcionamiento y eliminando el apartado de los filtros, convirtiéndolo en un mismo formulario.
    \item Añadir función de modificar eventos.
    \item Añadir función de borrar eventos.
    \item Arreglar el envío de mensaje al realizar una reserva.
    \item Añadir posibilidad de que el administrador añada propietarios nuevos.
    \item Añadir posibilidad al administrador de modificar los propietarios existentes.
    \item Arreglar la búsqueda de la página de reservas.
    \item Añadir un botón al consultar los eventos que permita mostrar y ocultar los eventos pasados.
    \item Arreglar los casos de uso tras la revisión realizada por el tutor.
    \item Rediseñar la página de consulta de eventos, para que sea mas intuitiva y más útil.
    \item Añadir una funcionalidad para que se puedan borrar más de un evento a la vez.
    \item Arreglar o mejorar pequeños detalles.
\end{itemize}

\subsection{Sprint 10 - 18/06/2020 - 26/06/2020 }
Realizar los últimos archivos de documentación que quedan por rellenar y completar o modificar algunos ya realizados.
\begin{itemize}
    \item Realizar el apartado de trabajos relacionados.
    \item Realizar el manual del programador.
    \item Realizar el manual del usuario.
    \item Mejoras sobre la documentación tras revisión del tutor.
\end{itemize}
\subsection{Sprint 11 - 01/07/2020 - 16/07/2020 }
Completar la aplicación con mejoras visuales y de funcionalidad.
\begin{itemize}
    \item Añadir una opción de seleccionar todos los eventos de la tabla para borrarlos a la vez.
    \item Añadir un filtro de fechas para la búsqueda en la tabla de auditorías.
    \item Permitir acceder a la aplicación como un usuario sin registrar y poder realizar la consulta de aulas y de eventos.
    \item Mejorar la pantalla de ayuda del mensaje al realizar acciones sobre el propietario.
    \item Permitir una búsqueda de un rango de fechas en la visualización de los eventos.
    \item Añadir más información al email que se envía al crear una reserva.
    \item Arreglar la reserva periódica que fallaba en algunos casos concretos.
    \item Actualizar la suscripción de Azure al haberse finalizado la de estudiante.
    \item Arreglar la consultas para ver las horas reservadas.
    \item Arreglar la modificación de la reserva al cambiar la hora a una que coincide con esta misma reserva.
\end{itemize}
\section{Estudio de viabilidad}
En esta sección trataré la viabilidad del proyecto, tanto en el ámbito económico como legal.
\subsection{Viabilidad económica}
En este apartado trataré de alcanzar el precio total del proyecto si fuera desarrollado en una empresa o comprado por alguien.\newline
\subsubsection{Coste de personal}
Para realizar este proyecto desde una vista de cliente, se ha necesitado un desarrollador de aplicaciones web para lograrlo. Por lo tanto el coste del personal se estima en lo siguiente:\newline
\subsubsection{Desarrollador}
Desarrollador: Suponiendo que acaba de comenzar su carrera como desarrollador web, el sueldo medio está entre 17.000 y 22.000. Suponemos un término medio de 20.000 para realizar los cálculos. Lo que supone a la empresa: 
    \begin{equation}
        20.000 \div 12 = 1667 bruto
    \end{equation}
    El trabajo se ha realizado de finales de Febrero a finales de Junio, lo que suponen 4 meses.
\tablaSmallSinColores{Coste Desarrollador}{p{6.4cm} p{2.15cm} p{8cm}}{costedesarrollador}{
  \multicolumn{1}{p{4.5cm}}{\textbf{Concepto}} & \textbf{Coste mensual{}}\\
 }
 {
  Salario mensual  & \multicolumn{1}{r}{1667}\\\hline
  Retención IRPF (10\%) & \multicolumn{1}{r}{166}\\\hline
  Costes seguridad social (23,60\%) & \multicolumn{1}{r}{393}\\\hline
  Cotización por formación profesional (0'6\%) &
  \multicolumn{1}{r}{10}\\\hline
  \textbf{Coste total (4 meses)}  & \multicolumn{1}{r}{7237}\\\hline
  }



\subsubsection{Coste hardware}
El coste del hardware utilizado se reduce al coste del portátil usado para desarrollar el proyecto. El precio del ordenador portátil (Asus X541U) con IVA fue de 600 y se supone una amortización de este mismo de 4 años. Por lo que el precio usado del portátil sería:\newline
\begin{equation}
    (600 \div(12 * 4))* 4 = 50
\end{equation}

\subsubsection{Coste despliegue}
El despliegue se ha realizado en Azure, donde se han ido realizando llamadas a la base de datos y ejecutando la aplicación subida. Esto se realizó con una suscripción a la cuenta de estudiante de Azure en la que se proporciona 100USD gratuitos, el uso de esta aplicación web para pruebas y mejoras ha supuesto un gasto de 100USD de los 100 disponibles, por lo que se agotó la suscripción de estudiante y hubo que mejorarla a una suscripción de pago por uso. Por lo que el gasto en este caso ha sido 110, teniendo en cuenta que ha sido durante 4 meses, el coste mensual se estima en 27.5. Pero este coste se aplicaría a una empresa durante este período de desarrollo. Si se quisiera mantener después aumentaría dependiendo del uso que se hiciera de esta aplicación y del plan de pago que se eligiera.

\subsubsection{Coste total}
\tablaSmallSinColores{Coste Total}{p{6.4cm} p{2.15cm} p{8cm}}{costetotal}{
  \multicolumn{1}{p{4.5cm}}{\textbf{Concepto}} & \textbf{Coste{}}\\
 }
 {
  Coste desarrollador  & \multicolumn{1}{r}{8944}\\
  Hardware & \multicolumn{1}{r}{50}\\\hline
  Despliegue & \multicolumn{1}{r}{110}\\\hline
  Internet & \multicolumn{1}{r}{26*4} \\\hline
  \textbf{Coste total (4 meses)}  & \multicolumn{1}{r}{9208}\\\hline
  }


\subsection{Viabilidad legal}

En cuanto al código, todo lo utilizado en este proyecto es de libre distribución y dominio público. Se ha utilizado Flask y librerías de Python públicas.\newline
En cuanto a los programas utilizados, todos son públicos también.

\tablaSmallSinColores{Dependencias del proyecto.}{l c}{dependencias}
{ \multicolumn{1}{l}{\textbf{Dependencia}} & \textbf{Licencia}\\}{ 
astroid & LGPL\\\hline
click & BSD\\\hline
colorama & BSD\\\hline
distlib & Python Software Foundation License\\\hline
dominate  & LGPLv3\\\hline
filelock  & Public Domain\\\hline
Flask  & BSD\\\hline
Flask-Bootstrap & BSD\\\hline
Flask-SQLAlchemy  & BSD\\\hline
Flask-WTF  & BSD\\\hline
importlib-metadata  & Apache Software License\\\hline
importlib-resources  & Apache Software License\\\hline
isort   & MIT\\\hline
itsdangerous   & BSD\\\hline
Jinja2   & BSD\\\hline
MarkupSafe   & BSD\\\hline
mysqlclient   & GPL\\\hline
pylint   & GPL\\\hline
pyodbc  & MIT\\\hline
requests   & Apache\\\hline
six   & MIT\\\hline
SQLAlchemy  & MIT\\\hline
virtualenv   & MIT\\\hline
visitor  & MIT\\\hline
Werkzeug  & BSD\\\hline
wrapt  & BSD\\\hline
WTForms   & BSD\\\hline
zipp   & MIT\\\hline

}

Una vez se han visto las licencias a las que están sometidas las dependencias del proyecto, se ha seguido las recomendaciones de \cite{licencias} para determinar que este proyecto está bajo la licencia GPL-3.0, ya que es la última versión de la licencia GPL sobre software libre con \textit{copyleft}, lo que permite que todas las versiones sean libres.