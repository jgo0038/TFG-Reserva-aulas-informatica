\capitulo{3}{Conceptos teóricos}

\subsection{Flask}
Es un framework en lenguaje Python que permite y facilita la creación de aplicaciones web con un comienzo sencillo y una escalabilidad muy alta. Utiliza la especificación Werkzeug y Jinja como motor en los templates.

\subsection{Werkzeug}
Es una biblioteca de aplicaciones web de WSGI (interfaz de puerta de enlace del servidor web), se utiliza internamente en Flask y hace las siguientes funciones:
\begin{itemize}
\item Depurador para inspeccionar rastros de la pila y el código fuente.
\item Objeto de solicitud para interactuar con encabezados, argumentos de consulta, datos de formulario, archivos y cookies.
\item Objeto de respuesta en la transmisión de datos.
\item Enrutamiento de URL.
\item Utilidades HTTP.
\end{itemize}

\subsection{HTML}
Las siglas hacen referencia a HyperText Markup Language, lo que viene a ser el lenguaje utilizado para la elaboración de las páginas web.

\subsection{Bootstrap}
Es una biblioteca de código abierto que proporciona estilos, plantillas y diseños ya creados para usar directamente en nuestras aplicaciones web.
 
\subsection{JavaScript}
Es un lenguaje de programación ligero e interpretado, orientado a objetos. En mi caso utilizado como lenguaje de scripting para páginas web. Se ha utilizado desde el lado del cliente proporcionando mejoras de las interfaz y de la comunicación con el servidor.

\subsection{MySQL}
Programa gestor de base de datos, de código abierto, en este caso utilizado para el comienzo y las pruebas del código en local.

\subsection{Azure}
Es un servicio de computación en la nube, de Microsoft, diseñado para construir, probar, desplegar y administrar aplicaciones y servicios mediante el uso de sus centros de datos. Proporciona muchas herramientas para facilitar el manejo de estas aplicaciones o servicios. En mi caso la he utilizado para desplegar mi aplicación web y mi base de datos. Ambas pueden ser gestionadas desde extensiones del \texit{VSCode}.

\subsection{Pyodbc}
Es un módulo de Python de código abierto que simplifica el acceso a las bases de datos ODBC. Usado para realizar la conexión entre la base de datos subida al servicio de Azure y nuestra aplicación web también desplegada.

