\capitulo{3}{Conceptos teóricos}

\subsection{Frameworks de desarrollo}
Un framework es un entorno o marco de trabajo estandarizado, con una estructura conceptual y tecnológica que puede servir de base para la organización y desarrollo de software \cite{wiki:framework}.\newline
En este proyecto he utilizado como framework Flask, que se explicará en las herramientas usadas.

\subsection{API REST}
"Una API REST es cualquier interfaz entre sistemas que utilice directamente una llamada de HTTP para obtener datos o indicar la ejecución de operaciones sobre los datos, en cualquier formato (XML, JSON, etc)". Esto permite que una aplicación puede interactuar con un recurso proporcionado en REST conociendo el identificador del recurso y la acción requerida para obtener la finalidad deseada \cite{wiki:REST}.\newline
En nuestro caso se ha utilizado el endpoint de Microsoft Graph para acceder a la API REST de Outlook, para ello hubo que realizar lo siguiente:\newline
Primeramente se utiliza la API de Microsoft para  para realizar el login en la aplicación mediante una cuenta de Outlook. Utiliza OAuth 2.0, que necesita el token recibido al registrar la aplicación en Azure Active Directory y nos devuelve un nuevo token. Una vez lo recibimos podemos hacer uso del endpoint de Microsoft Graph mediante este token de OAuth recibido.
\newline
Una vez se dispone de este acceso, se realizan llamadas mediante el punto de conexión Microsoft Graph: punto de conexión común para acceder a la API REST que permite el acceso y la modificación a todas las características de una cuenta Microsoft O365. Dentro de este punto de conexión se encuentran las llamadas al resto de las APIs de Microsoft para lo que necesitemos (calendarios, usuarios, grupos de calendarios, correo).


\subsection{APIs de validación}
Permite implementar el inicio de sesión que utiliza una aplicación en tu propio proyecto, de forma que si existe esa cuenta permite el acceso a la dirección que se haya especificado. Para que este proceso funcione, Microsoft utiliza OAuth v2.0, que sigue un flujo de autorización y autenticación para acceder a la aplicación y redireccionar al usuario a la página indicada en Azure Active Directory. Este proceso devuelve un token de acceso a la aplicación que sirve para obtener acceso a los recursos protegidos mediante el punto de conexión de la API utilizada y otro token para refrescar el código de acceso anterior en caso de que caduque si el inicio de sesión es correcto. Un token es un código de letras y números único que se necesita para realizar el resto de llamadas a la API en la que nos hemos logeado para obtener información sobre la sesión que hemos iniciado.

\subsection{APIs de calendarios}
La API de un calendario nos permite guardar eventos dentro de los calendarios que nos proporciona, borrar, modificar o compartir estos eventos y calendarios. Mediante llamadas a esta API se pueden acceder a los calendarios, recuperar los eventos y realizar las distintas modificaciones sobre ellos. Esta API que hemos usado pertenece a la API REST de Outlook y accedemos a ella mediante el punto de acceso de Microsoft Graph como se ha explicado anteriormente.

\subsection{APIs de conectividad de BD}
Es una capa intermedia de acceso a las bases de datos que hace posible las acciones \textit{CRUD} a cualquier dato desde la aplicación que desarrollamos \cite{wiki:ODBC}. En en el presente trabajo utilizamos \textit{pyodbc}: Es un módulo de Python de código abierto que simplifica el acceso a las bases de datos ODBC. Usado para realizar la conexión entre la base de datos y la aplicación web desarrollada, alojadas ambas en Azure.

