\capitulo{3}{Conceptos teóricos}

\subsection{Frameworks de desarrollo}
Un framework es un entorno o marco de trabajo estandarizado, con una estructura conceptual y tecnológica que puede servir de base para la organización y desarrollo de software \cite{wiki:framework}.\newline
En este proyecto he utilizado como framework Flask, que es un 'micro' framework en lenguaje Python que permite y facilita la creación de aplicaciones web con un comienzo sencillo y una escalabilidad muy alta. Utiliza la especificación Werkzeug y Jinja como motor en los templates.\newline
Que sea un framework de tipo 'micro' significa que al instalar Flask tenemos las herramientas necesarias para crear una aplicación web funcional \cite{FlaskDefinicion}.


\subsection{Werkzeug}
Es una biblioteca de aplicaciones web de WSGI (interfaz de puerta de enlace del servidor web), se utiliza internamente en Flask y hace las siguientes funciones:
\begin{itemize}
\item Depurador para inspeccionar rastros de la pila y el código fuente.
\item Objeto de solicitud para interactuar con encabezados, argumentos de consulta, datos de formulario, archivos y cookies.
\item Objeto de respuesta en la transmisión de datos.
\item Enrutamiento de URL.
\item Utilidades HTTP.
\end{itemize}

\subsection{API REST}
Una API REST es cualquier interfaz entre sistemas que utilice directamente una llamada de HTTP para obtener datos o indicar la ejecución de operaciones sobre los datos, en cualquier formato (XML, JSON, etc). Esto permite que una aplicación puede interactuar con un recurso proporcionado en REST conociendo el identificador del recurso y la acción requerida para obtener la finalidad deseada \cite{wiki:REST}.

\subsection{APIs de conectividad de BD}
Es un estándar de acceso a las bases de datos que hace posible el acceder a cualquier dato desde la aplicación que desarrollamos \cite{wiki:ODBC}. En mi caso utilizo \textit{pyodbc}:\newline
Es un módulo de Python de código abierto que simplifica el acceso a las bases de datos ODBC. Usado para realizar la conexión entre la base de datos subida al servicio de Azure y nuestra aplicación web también desplegada.

