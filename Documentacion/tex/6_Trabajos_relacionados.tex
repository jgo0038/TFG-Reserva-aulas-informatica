\capitulo{6}{Trabajos relacionados}

Esta aplicación web tiene similitudes con todas las aplicaciones de reserva de espacios, pero en lo que distingo mi aplicación principalmente es en el uso de Outlook tanto para la validación del usuario como para la gestión de los calendarios usando Outlook calendar.\newline
En cuanto a funcionalidades, mirando aplicaciones similares de reservas, la principal diferencia a destacar es que ésta está realizada para la universidad o entidades similares, ya que tengo en cuenta distintos edificios en los que se manejan distintas aulas y dichas aulas tienen características y se proporcionan filtros concretos de aulas de universidad.\newline Por ejemplo, aplicaciones similares que he encontrado:
\begin{itemize}
    \item \textit{BookedApp}\newline 
    Es una aplicación que se dedica a la reserva de sitios individuales, no aulas, de biblioteca, mi aplicación web gestiona más de un espacio pero no por sitios individualmente, sino reservas de aulas completas.
    \item  \textit{Booked Scheduler – Reserva de aulas}\href{https://valentingom.wordpress.com/2016/07/10/booked-scheduler-reservas/}{Enlace}\cite{BookedScheduler} \newline
    Es una aplicación más desarrollada que permite reservas de aulas de forma similar a la nuestra, no está sincronizado con Outlook y todo se lleva en base de datos, sin embargo como ventaja suya podemos apreciar los eventos representados en instancias de calendarios.
\end{itemize}
(\textit{BookedApp}) que se dedica a la reserva de sitios de biblioteca, mi aplicación web gestiona más de un espacio pero no por sitios individualmente, sino reservas de aulas completas.\newline

